\documentclass[10pt,a4paper]{article}
\usepackage[utf8]{inputenc}
\usepackage[french]{babel}
\usepackage{amsmath}
\usepackage{amsfonts}
\usepackage{amssymb}
\author{Pierre Gimalac \& Alexandre Moine}
\title{Rapport de projet math-info}
\begin{document}
\maketitle

\section{CTL, une logique temporelle}
CTL (pour Computation Tree Logic) est une logique temporelle.

\subsection{Grammaire}
La grammaire de CTL est définie de la manière suivante:
\begin{align*}
\phi &::= \bot \mid \top \mid p \mid \neg p \mid \phi\land\phi \mid \phi\lor\phi \mid \\
&\quad \mbox{AX }\phi \mid \mbox{EX }\phi \mid
\mbox{A }\phi \mbox{ W } \phi \mid \mbox{E }\phi \mbox{ W } \phi \mid
\mbox{A }\phi \mbox{ U } \phi \mid \mbox{E }\phi \mbox{ U } \phi
\end{align*}

On peut définir d'autres opérateurs (AF, EF, AG, EG, NEG).

\section{Model-checking de CTL: parcours de graphe}

\section{Model-checking de CTL: jeux faibles de parité}
\end{document}
\documentclass[10pt,a4paper]{article}
\usepackage[utf8]{inputenc}
\usepackage[french]{babel}
\usepackage{amsmath}
\usepackage{amsfonts}
\usepackage{amssymb}
\usepackage{stmaryrd}
\usepackage{enumerate}
\usepackage{algorithm}
\usepackage{algorithmic}
\usepackage{hyperref}

\author{Pierre Gimalac \& Alexandre Moine\\\small{Encadré par François Laroussinie}}
\title{Rapport de Projet\\Mathématiques-Informatique}
\begin{document}
\maketitle

\section{CTL, une logique temporelle}
\subsection{Définition}
CTL (pour \textit{Computation Tree Logic}) est une logique temporelle. Elle permet d'exprimer des contraintes logiques (CTL contient la logique propositionnelle usuelle) et temporels.
Elle permet par exemple d'exprimer des propriétés comme ``il existe une exécution telle que la variable \textit{a} soit toujours vraie''.

\subsection{Grammaire}
\label{op}
La grammaire de CTL est définie de la manière suivante :
\begin{align*}
\phi &::= \bot \mid \top \mid p \mid \neg \phi \mid \phi\land\phi \mid \phi\lor\phi \mid \\
&\quad \mbox{AX }\phi \mid \mbox{EX }\phi \mid
\mbox{A }\phi \mbox{ W } \phi \mid \mbox{E }\phi \mbox{ W } \phi \mid
\mbox{A }\phi \mbox{ U } \phi \mid \mbox{E }\phi \mbox{ U } \phi
\end{align*}

Ici, $p$ représente une proposition atomique d'un ensemble $\Omega$.

On définit aussi d'autres opérateurs utiles :
\begin{itemize}
	\item $\mbox{AF } \phi := \mbox{A } \top \mbox{ U } \phi$
	\item $\mbox{EG } \phi := \neg (\mbox{AF } (\neg \phi))$
	\item $\mbox{EF } \phi := \mbox{E } \top \mbox{ U } \phi$
	\item $\mbox{AG } \phi := \neg (\mbox{EF } (\neg \phi))$
\end{itemize}

\subsection{Sémantique de CTL}
Les modèles de CTL sont appelés des \emph{structures de Kripke}.

\subsubsection{Structures de Kripke}
Une structure de Kripke sera ici identifiée à la donnée de $(Q,T,q_0,l)$ où:
\begin{itemize}
\item Le couple $(Q,T, q_0)$ est un graphe orienté : $Q$ est un ensemble d'états, $T \subseteq Q \times Q$ est la relation de transition et $q_0 \in Q$ représente l'état de départ.
\item $l : Q \to \mathcal{P}(\Omega)$ est une fonction d'étiquetage.
\end{itemize}
\bigskip
On ne considère ici que des cas où $Q$ est fini et $T$ est totale.

\subsubsection{Exécution}
Soit $\mathcal{A} = (Q,T,q_0,l)$ une structure de Kripke.
On dira que $\sigma : \mathbb{N} \to Q$ est une \emph{exécution} de $\cal{A}$ (ne partant pas forcément de l'état initial) si et seulement si $\forall i \in \mathbb{N}, (\sigma (i), \sigma (i+1)) \in T$. 
\\
On note $Exec(\mathcal{A})$ l'ensemble des exécutions sur $\mathcal{A}$.

\subsubsection{Satisfaction}
Soit $\mathcal{A} = (Q,T,q_0,l)$ une structure de Kripke et $\phi$ une formule de CTL.\\
On dira que $q$ vérifie $\phi$ dans $\mathcal{A}$ si et seulement si $A, q \vDash \phi$. On omettra souvent la donnée de $\mathcal{A}$ dans notre écriture.\\
On définit $q \vDash \phi$ par induction sur $\phi$:\\
\\
\begin{tabular}{lcl}
$q \nvDash \bot$ &&\\
$q \vDash \top$ &&\\
$q \vDash p$ &ssi& $p \in l (q)$\\
$q \vDash \neg \psi$ &ssi& $q \nvDash \psi$\\
$q \vDash \psi_1 \land \psi_2$ &ssi& $q \vDash \psi_1$ et $q \vDash \psi_2$\\
$q \vDash \psi_1 \lor \psi_2$ &ssi& $q \vDash \psi_1$ ou $q \vDash \psi_2$\\
$q \vDash \mbox{EX } \psi$ &ssi& $\exists \sigma \in Exec_\mathcal{A}(q)$, $\sigma(1) \vDash \psi$\\
$q \vDash \mbox{AX } \psi$ &ssi& $\forall \sigma \in Exec_\mathcal{A}(q)$, $\sigma(1) \vDash \psi$\\

$q \vDash \mbox{E } \psi_1 \mbox{ U } \psi_2$ &ssi& $\exists \sigma \in Exec_\mathcal{A}(q)$, $\exists k \in \mathbb{N}$, $\sigma(k)\vDash \psi_2$ $\land$ $\forall j \in \llbracket 0; k\llbracket, \sigma(j)\vDash \psi_1$\\

$q \vDash \mbox{A } \psi_1 \mbox{ U } \psi_2$ &ssi& $\forall \sigma \in Exec_\mathcal{A}(q)$, $\exists k \in \mathbb{N}$, $\sigma(k)\vDash \psi_2$ $\land$ $\forall j \in \llbracket 0; k\llbracket, \sigma(j)\vDash \psi_1$\\

$q \vDash \mbox{E } \psi_1 \mbox{ W } \psi_2$ &ssi& $\exists \sigma \in Exec_\mathcal{A}(q)$, soit $\forall j \in \mathbb{N}$, $\sigma(j) \vDash \psi_1$\\
& & soit $\exists k\in \mathbb{N}$: $\sigma(k)\vDash \psi_2 \land \forall j \in \llbracket i; k \llbracket$: $\sigma(j) \vDash \psi_1$\\

$q \vDash \mbox{A } \psi_1 \mbox{ W } \psi_2$ &ssi& $\forall \sigma \in Exec_\mathcal{A}(q)$, soit $\forall j \in \mathbb{N}$, $\sigma(j) \vDash \psi_1$\\
& & soit $\exists k\in \mathbb{N}$: $\sigma(k)\vDash \psi_2 \land \forall j \in \llbracket i; k \llbracket$: $\sigma(j) \vDash \psi_1$\\
\end{tabular}

\bigskip

La sémantique de CTL se comprend aussi bien en français:

\begin{tabular}{lcl}
$q \vDash \top$ && est toujours vrai\\
$q \vDash \bot$ && n'est jamais vrai\\
$q \vDash p$ &ssi& $p$ est une étiquette de $q$\\
$q \vDash \neg \psi$ &ssi& $q$ ne vérifie pas $\phi$\\
$q \vDash \psi_1 \land \psi_2$ &ssi& $q$ vérifie $\phi_1$ et $q$ vérifie $\phi_2$\\
$q \vDash \psi_1 \lor \psi_2$ &ssi& $q$ vérifie $\phi_1$ ou $q$ vérifie $\phi_2$\\
$q \vDash \mbox{EX } \psi$ &ssi& il existe un successeur de $q$ vérifiant $\psi$\\
$q \vDash \mbox{AX } \psi$ &ssi& tous les successeurs de $q$ vérifient $\psi$\\
$q \vDash \mbox{E } \psi_1 \mbox{ U } \psi_2$ &ssi& il existe un chemin à partir de $q$ tel que $\psi_1$\\&& est vrai jusqu'à ce que $\psi_2$ le soit (et il le sera un jour)\\

$q \vDash \mbox{A } \psi_1 \mbox{ U } \psi_2$ &ssi& pour tout chemin partant de $q$, $\psi_1$ est vrai \\&& jusqu'à ce que $\psi_2$ le soit (et il le sera un jour)\\

$q \vDash \mbox{E } \psi_1 \mbox{ W } \psi_2$ &ssi& il existe un chemin à partir de $q$ tel que $\psi_1$\\&& est vrai jusqu'à ce que, peut-être, $\psi_2$ le soit\\

$q \vDash \mbox{A } \psi_1 \mbox{ W } \psi_2$ &ssi& pour tout chemin partant de $q$, $\psi_1$ est vrai \\&& jusqu'à ce que, peut-être, $\psi_2$ le soit\\
\end{tabular}

Cette sémantique permet aussi de mieux comprendre les opérateurs définis en Section \ref{op}:

\begin{tabular}{lcl}
$q \vDash \mbox{AF } \phi$ & ssi & tous les chemins depuis $q$ permettent d'atteindre un état vérifiant $\phi$. \\
$q \vDash \mbox{EG } \phi$ & ssi & il existe un chemin depuis $q$ tel que tous les états de ce chemin satisfassent $\phi$.\\
$q \vDash \mbox{EF } \phi$ & ssi & il existe un chemin depuis $q$ permettent d'atteindre un état vérifiant $\phi$. \\
$q \vDash \mbox{AG } \phi$ & ssi & tous les états atteignables depuis $q$ vérifient $\phi$.\\
\end{tabular}

\subsection{Négation}
On peut définir la fonction $desc\_neg:CTL \to CTL$ qui permet de transformer une formule de CTL en formule équivalente où toutes les négations sont sur les littéraux :
\begin{align*}
&desc\_neg(\top) = \top \quad desc\_neg(\bot) = \bot\\
&desc\_neg(p) = p \\
&desc\_neg(\neg \phi) = neg(\phi)\\
&desc\_neg(a \lor b) = desc\_neg(a) \lor desc\_neg(b)\\
&desc\_neg(a \land b) = desc\_neg(a) \land desc\_neg(b)\\
&desc\_neg(\mbox{EX } \phi) = \mbox{EX }desc\_neg(\phi)\\
&desc\_neg(\mbox{AX } \phi) = \mbox{AX }desc\_neg(\phi)\\
&desc\_neg(\mbox{E } \phi \mbox{ U } \psi) =\mbox{E } desc\_neg(\phi) \mbox{ U } desc\_neg(\psi)\\
&desc\_neg(\mbox{A } \phi \mbox{ U } \psi) =\mbox{A } 
desc\_neg(\phi) \mbox{ U } desc\_neg(\psi)\\
&desc\_neg(\mbox{E } \phi \mbox{ W } \psi) =\mbox{E } desc\_neg(\phi) \mbox{ W } desc\_neg(\psi)\\
&desc\_neg(\mbox{A } \phi \mbox{ W } \psi) =\mbox{A } desc\_neg(\phi) \mbox{ W } desc\_neg(\psi)
\end{align*}

\begin{align*}
&neg(\top) = \bot \quad neg(\bot) = \top\\
&neg(p) = \neg p \quad neg(\neg \phi) = desc\_neg(\phi)\\
&neg(a \lor b) = neg(a) \land neg(b)\\
&neg(a \land b) = neg(a) \lor neg(b)\\
&neg(\mbox{EX } \phi) = \mbox{AX }neg(\phi)\\
&neg(\mbox{AX } \phi) = \mbox{EX }neg(\phi)\\
&neg(\mbox{E } \phi \mbox{ U } \psi) =\mbox{A } (neg(\psi)) \mbox{ W } (neg(\phi) \land neg(\psi))\\
&neg(\mbox{A } \phi \mbox{ U } \psi) =\mbox{E } (neg(\psi)) \mbox{ W } (neg(\phi) \land neg(\psi))\\
&neg(\mbox{E } \phi \mbox{ W } \psi) =\mbox{A } (neg(\psi)) \mbox{ U } (neg(\phi) \land neg(\psi))\\
&neg(\mbox{A } \phi \mbox{ W } \psi) =\mbox{E } (neg(\psi)) \mbox{ U } (neg(\phi) \land neg(\psi))
\end{align*}

\section{Model-checking de CTL: jeux faibles de parité}
Le but ici est de parvenir à créer un jeu à partir d'une formule et d'une structure de Kripke où deux joueurs (nommés Ève et Adam) s'affrontent, et Ève gagne si et seulement si la formule est vraie dans la structure donnée.

\subsection{Jeux de parité}
On appelle \emph{jeu de parité} un quadruplet $G = (V_E, V_A, R, c)$ où:
\begin{itemize}
\item $V_E$ (respectivement $V_A$) est l'ensemble des états jouables par Ève (respectivement Adam). On demande de plus que $V_E \cap V_A = \emptyset$. On note $V = V_E \sqcup V_A$.
\item $R \subseteq V \times V$ est une relation de transition.
\item $c : V \to (\mathbb{N} \cup \{\bot\})$ est une fonction qui associe une couleur ou $\bot$ à chaque état.
\end{itemize}

\subsubsection{Victoire}
Une partie d'un jeu est une fonction $\omega : V' \to V, V' \subseteq V$ telle que $\forall v \in V', (v,\omega(v)) \in R$.
Une partie à partir d'un état $q_0 \in V$ est gagnée par:
\begin{itemize}
\item Ève si est seulement si:
	\begin{itemize}
	\item Soit Adam est bloqué : $\exists n \in \mathbb{N}, \omega^n (q_0) \in V_A \land \omega^{n+1} (q_0) \notin V'$.
	\item Soit Ève visite infiniment souvent des états uniquement pairs: $\exists n \in \mathbb{N}, \forall m \geq n, (\omega^m(q_0) \in V_E \land c(\omega^m(q_0)) \neq \bot )\implies c(\omega^m(q_0)) = 0 \mod 2$.
	\end{itemize}
\item Adam si est seulement si:
\begin{itemize}
	\item Soit Ève est bloquée : $\exists n \in \mathbb{N}, \omega^n (q_0) \in V_E \land \omega^{n+1} (q_0) \notin V'$.
	\item Soit Adam visite infiniment souvent des états uniquement impairs: $\exists n \in \mathbb{N}, \forall m \geq n, (\omega^m(q_0) \in V_E \land c(\omega^m(q_0)) \neq \bot )\implies c(\omega^m(q_0)) = 1 \mod 2$.
\end{itemize}
\end{itemize}

\subsubsection{Stratégie}
On appelle une stratégie pour Ève (respectivement Adam) une fonction $\rho : V_E' \to V$ (respectivement $\rho : V_A' \to V$), $V_E' \subseteq V_E$ et $V_A' \subseteq V_A$, telle que pour chaque $v$ dans $V_E'$ (respectivement $V_A'$), $(v, \rho(v)) \in R$.

On appelle stratégie \emph{induite} par $\rho : V_E' \to V$ et $\rho' : V_A' \to V$, la fonction $\omega : (V_E' \sqcup V_A' \to V)$ définie par:
\\
$$\omega(v) = \left \{
\begin{array}{rcl}
\rho(v) & si & v \in V_E'\\
\rho'(v) & si & v \in V_A'\\
\end{array}
\right .$$
Une stratégie $\rho$ est dite gagnante pour Ève (respectivement Adam) si et seulement si pour toute stratégie $\rho'$ pour Adam (respectivement Ève), la partie induite par $\rho$ et $\rho'$ est gagnante pour Ève (respectivement Adam).

\subsubsection{Solution}
Les jeux de parité sont \emph{déterminés}, c'est-à-dire que chaque état est gagnant soit pour un joueur, soit pour l'autre.

\subsubsection{Résolution}
La résolution des jeux de parité est un problème dans $NP \cap coNP$, et on ne sait pas s'il existe un algorithme déterministe polynomial qui le résout.

\subsection{Jeux faibles de parité}
\label{fpg}

On dit qu'un jeu de parité $G = (V_E,V_A,R,c)$ est \emph{faible}  si et seulement si il existe $V_1$, ..., $V_k$ une partition de $V = V_E \sqcup V_A$ telle que:
\begin{enumerate}[a)]
\item La couleur soit la même pour tous les états d'une classe de la partition : $\forall i \in \llbracket 1 ; k \rrbracket$, $\forall x, y \in V_i$, $(c(x) \neq \bot \land c(y) \neq \bot)  \implies c(x) = c(y)$.
\item La partition représente une convergence du jeu :\\
$\forall (x, y) \in R$, $x\in V_i, y \in V_j \Rightarrow i \leq j$.
\end{enumerate}

\subsection{Résolution}
Les jeux faibles de parité se résolvent en temps polynomial comme nous le verrons en Section \ref{algofpg}.

\subsection{Algorithme de construction du jeu}
Nous allons maintenant définir un algorithme qui permet de construire un jeu faible de parité à partir d'une formule de CTL et d'une structure de Kripke. Nous allons ensuite montrer que ce jeu est \emph{faible} et qu'Ève gagne si et seulement si la formule est vraie dans la structure. Plus précisément, nous allons donner ici la fonction de transition du jeu.

Soit $\phi$ une formule de CTL, $SF(\phi)$ l'ensemble de ses sous-formules et $K = (Q,T,q_0,l)$  une structure de Kripke .
On va construire un jeu de parité $G = (V_E,V_A,R,c)$ où:
\begin{itemize}
	\item $V = Q \times SF(\phi)$
	\item $V_A = \{(i, \top ) \in V\} \cup \{(i, \phi) \in V, \phi = \phi_1 \land \phi_2 \mid \mbox{AX } \phi_1 \}$
	\item $V_E = V \backslash V_A$
	\item $c(s,v) = \left \{
	\begin{array}{rcl}
		2 & si & v = \mbox{E } \phi_1 \mbox{ W } \phi_2\text{ ou }\mbox{A } \phi_1 \mbox{ W } \phi_2\\
		1 & si & v = \mbox{E } \phi_1 \mbox{ U } \phi_2\text{ ou }\mbox{A } \phi_1 \mbox{ U } \phi_2\\
		\bot && \text{sinon}
	\end{array}
	\right .$
\end{itemize}

\bigskip

On peut maintenant définir $R$, la relation de transition, en définissant la fonction $succ : V \to P(V)$ qui à un état associe la liste de ses successeurs.\\

On définit $succ$:
\begin{itemize}
\item$ succ(s, \bot) = succ(s, \top) = \emptyset$
\item $succ(s, p) = \{(s,\top)\}$ si $p \in l(s)$, $\{(s,\bot)\}$ sinon
\item $succ(s, \neg p) = \{(s,\bot)\}$ si $p \in l(s)$, $\{(s,\top)\}$ sinon
\item $succ(s,\phi \land \psi) = succ(s, \phi \lor \psi) = \{(s,\phi), (s,\psi) \}$
\item $succ(s,EX \phi) = succ(s, AX \phi) = \{ (s', \phi) \mid (s,s') \in T \} $
\item $succ(s, \mbox{E } \phi \mbox{ U } \psi) = \{ (s,\psi), (s, \phi \land \mbox{EX }  (\mbox{E } \phi \mbox{ U } \psi)) \}$
\item $succ(s, \mbox{A } \phi \mbox{ U } \psi) = \{ (s,\psi), (s, \phi \land \mbox{AX }  (\mbox{A } \phi \mbox{ U } \psi)) \}$
\item $succ(s, \mbox{E } \phi \mbox{ W } \psi) = \{ (s,\psi), (s, \phi \land \mbox{EX }  (\mbox{E } \phi \mbox{ W } \psi)) \}$
\item $succ(s, \mbox{A } \phi \mbox{ W } \psi) = \{ (s,\psi), (s, \phi \land \mbox{AX } (\mbox{A } \phi \mbox{ W } \psi)) \}$
\end{itemize}

\subsubsection{Jeu faible}
Montrons que le jeu défini par $succ$ avec comme état initial $start = (q_0, \phi)$.
 est un jeu \emph{faible}, donc qu'il existe une partition finie de $V$ telle que a) et b) (définies en Section \ref{fpg}) soient vraies.

On identifie le jeu à un graphe et on va montrer que la partition $V_1, \dots, V_k$ des composantes fortement connexes dans un ordre topologique satisfait les conditions. On note que les CFC forment un graphe acyclique.

\begin{enumerate}[a)]
\item $\forall i \in \llbracket 1 ; k \rrbracket$, $\forall x,y \in V_i$ tel que $x \neq y$ et $c(x) \neq \bot$ et $c(y) \neq \bot$. Si $V_i$ ne contient qu'un seul état, le résultat est vrai. Sinon, $V_i$ est une CFC non triviale et, à la vue de la définition de $succ$ elle est de la forme suivante : $\{ \phi; \mbox{EX }\phi \}$ avec $\mbox{X } \psi_1 \mbox{ Y } \psi_2$ et $ X=A$ ou $X =E$ et $Y = U$ ou $Y=W$. Donc le résultat est vrai.

\item est trivialement vraie: les $V_i$ sont ordonnées par ordre topologique.
\end{enumerate}

\subsubsection{Jeu correct}
Montrons maintenant qu'Ève a une stratégie gagnante dans $G$ si, et seulement si, $q_0 \vDash \phi$.

\paragraph{$\implies$}
On suppose qu'Ève a une stratégie gagnante à partir de $(q_0, \phi)$. On va montrer par récurrence sur $\phi$ que $q_0 \vDash \phi$.

\begin{itemize}
\item $\phi = p$. C'est à Ève de jouer et $succ(q_0,\phi) = \{(q_0, \top)\}$ ou $succ(q_0,\phi) = \{(q_0, \bot)\}$. Comme Ève a une stratégie gagnante, c'est qu'elle n'est pas bloquée dans le prochain état. Donc Adam l'est, donc $succ(q_0,\phi) = \{(q_0, \top)\}$, donc $p \in l(q_0)$ donc, $q_0 \vDash \phi$.

\item $\phi = \neg p$. C'est à Ève de jouer et $succ(q_0,\phi) = \{(q_0, \top)\}$ ou $succ(q_0,\phi) = \{(q_0, \bot)\}$. Comme Ève a une stratégie gagnante, c'est qu'elle n'est pas bloquée dans le prochain état. Donc Adam l'est, donc $succ(q_0,\phi) = \{(q_0, \top)\}$, donc $p \notin l(q_0)$ donc, $q_0 \vDash \phi$.

\item $\phi = \psi_2 \lor \psi_2$. On a donc $succ(q_0,\phi) = \{(q_0, \psi_1); (q_0, \psi_2) \}$ et c'est à Ève de jouer. Comme Ève a une stratégie gagnante, elle choisit dans quel état aller, mettons $(q_0, \psi_1)$. Elle a donc une stratégie gagnante à partir de $(q_0, \psi_1)$, donc par hypothèse de récurrence, $q_0 \vDash \psi_1$. Donc $q_0 \vDash \phi$. Il en va de même si Ève choisit d'aller en $(q_0, \psi_2)$.

\item $\phi = \psi_2 \land \psi_2$. On a donc $succ(q_0,\phi) = \{(q_0, \psi_1); (q_0, \psi_2) \}$ et c'est à Adam de jouer. Comme Ève a une stratégie gagnante, quel que soit l'état que choisit Adam, elle a une stratégie gagnante à partir de ce nouvel état, donc par hypothèse de récurrence, $q_0 \vDash \psi_1$ et $q_0 \vDash \psi_2$. Donc $q_0 \vDash \phi$.

\item $\phi = \mbox{EX } \psi$. Donc $succ(q_0, \phi) = \{ (q, \psi) \mid (q_0,q) \in T \} $ et c'est à Ève de jouer. Comme elle a une stratégie gagnante, il existe un successeur $q$ de $q_0$ tel qu'elle ait une stratégie gagnante à partir de $q$ et donc par récurrence que $q \vDash \psi$. Donc $q_0 \vDash \phi$.

\item $\phi = \mbox{AX } \psi$. Donc $succ(q_0, \phi) = \{ (q, \psi) \mid (q_0,q) \in T \} $ et c'est à Adam de jouer. Comme Ève a une stratégie gagnante, pour tous les successeurs $q$ de $q_0$, elle une stratégie gagnante à partir de $q$ et donc par récurrence que $q \vDash \psi$. Donc $q_0 \vDash \phi$.

\item $\phi = \mbox{E } \psi_1 \mbox{ U } \psi_2$ ou $\phi = \mbox{E } \psi_1 \mbox{ W } \psi_2$. Donc $succ(\phi) = \{ (s,\psi_2), (s, \psi_1 \land \mbox{EX } \phi) \}$. C'est à Ève de jouer et elle a une stratégie gagnante, deux cas sont possibles :
\begin{itemize}
	\item Soit Ève choisit $(s,\psi_2)$, et par hypothèse de récurrence $q_0 \vDash \psi_2$ et donc $q_0 \vDash \phi$.
	\item Soit Ève choisit $(s, \psi_1 \land \mbox{EX } \phi)$. Donc, par récurrence, $s \vDash \psi_1$. De plus, Ève a une stratégie gagnante à partir de $(s,\mbox{EX } \phi)$ et
	elle va répéter ce choix sur les successeurs de $s$ et il y a deux choix:
	\begin{itemize}
		\item Soit il existe un état $(q,\psi_2)$ qu'Ève va choisir, et donc, par récurrence, $q \vDash \psi_2$ et comme on a vu que pour tous états $q'$ sur le chemin entre $s$ et $q$, $q' \vDash \psi_1$, on a bien que $q_0 \vDash \phi$.
		\item Soit un tel état n'existe pas et donc Ève "boucle" sur un cycle composé avec infiniment souvent des états de la forme $(q,\phi)$. Comme elle a une stratégie gagnante, c'est que $c(\phi) = 0 \mod 2$ et donc que $\phi = \mbox{E } \psi_1 \mbox{ W } \psi_2 $, et donc, comme pour tous les états sur $q$ sur le cycle, on a vu que $q \vDash \psi_1$, on a bien $q_0 \vDash \phi$.
	\end{itemize}
\end{itemize}

\item $\phi = \mbox{A } \psi_1 \mbox{ U } \psi_2$ ou $\phi = \mbox{A } \psi_1 \mbox{ W } \psi_2$. Donc $succ(\phi) = \{ (s,\psi_2), (s, \psi_1 \land \mbox{AX } \phi) \}$. C'est à Ève de jouer et elle a une stratégie gagnante, deux cas sont possibles :
\begin{itemize}
	\item Soit Ève choisit $(s,\psi_2)$, et par hypothèse de récurrence $q_0 \vDash \psi_2$ et donc $q_0 \vDash \phi$.
	\item Soit Ève choisit $(s, \psi_1 \land \mbox{AX } \phi)$. Donc, par récurrence, $s \vDash \psi_1$. De plus, Ève a une stratégie gagnante à partir de $(s,\mbox{AX } \phi)$.
	Ève va répéter ce choix sur tous les successeurs de $s$ et il y a deux choix:
	\begin{itemize}
		\item Soit il existe un état $(q,\psi_2)$ qu'Ève va choisir, et donc, par récurrence, $q \vDash \psi_2$ et comme on a vu que pour tous états $q'$ sur le chemin entre $s$ et $q$, $q' \vDash \psi_1$, on a bien que $q_0 \vDash \phi$.
		\item Soit un tel état n'existe pas et donc Adam force Ève à "boucler" sur un cycle composé avec infiniment souvent des états de la forme $(q,\phi)$. Comme elle a une stratégie gagnante, c'est que $c(\phi) = 0 \mod 2$ et donc que $\phi = \mbox{A } \psi_1 \mbox{ W } \psi_2 $, et donc, comme pour tous les états sur $q$ sur le cycle, on a vu que $q \vDash \psi_1$, on a bien $q_0 \vDash \phi$.
	\end{itemize}
\end{itemize}
\end{itemize}

\paragraph{$\Longleftarrow$}
On suppose que $q_0 \vDash \phi$. On doit en déduire une stratégie gagnante pour Ève.\\
On procède par récurrence sur $\phi$ :
\begin{itemize}
\item $\phi = p$, c'est à Ève de jouer et comme $q_0 \vDash \phi$, cela veut dire que $p \in l(q_0)$, donc $succ(q_0,\phi) = \{(q_0, \top)\}$, donc Adam est bloqué, donc Ève gagne.
\item $\phi = \neg p$, c'est à Ève de jouer et comme $q_0 \vDash \phi$, cela veut dire que $p \notin l(q_0)$, donc $succ(q_0,\phi) = \{(q_0, \top)\}$, donc Adam est bloqué, donc Ève gagne.

\item $\phi = \psi_2 \lor \psi_2$. On a donc $succ(q_0,\phi) = \{(q_0, \psi_1); (q_0, \psi_2) \}$ et c'est à Ève de jouer. Comme $q_0 \vDash \phi$, $q_0 \vDash \psi_1$ ou $q_0 \vDash \psi_2$. Si, par exemple, $q_0 \vDash \psi_1$, par hypothèse de récurrence, on a qu'Ève a une stratégie gagnante à partir de l'état $(q_0, \psi_1)$. Ève peut donc choisir d'aller dans cet état, et elle gagne. On procède de la même façon si $(q_0, \psi_2)$.

\item $\phi = \psi_2 \land \psi_2$. On a donc $succ(q_0,\phi) = \{(q_0, \psi_1); (q_0, \psi_2) \}$ et c'est à Adam de jouer. Comme $q_0 \vDash \phi$, $q_0 \vDash \psi_1$ et $q_0 \vDash \psi_2$. On a donc par hypothèse de récurrence qu'Ève a une stratégie gagnante à partir de l'état $(q_0, \psi_1)$ et à partir de l'état $(q_0, \psi_2)$. Donc, quel que soit le coup que va jouer Adam, Ève a une stratégie gagnante.

\item $\phi = \mbox{EX } \psi$. Donc $succ(q_0, \phi) = \{ (q, \psi) \mid (q_0,q) \in T \} $ et c'est à Ève de jouer. Comme $q_0 \vDash \phi$, il existe un successeur $q$ de $q_0$ tel que $q \vDash \psi$. Donc, par hypothèse de récurrence, il existe une stratégie gagnante pour Ève à partir de $(q, \psi)$. Comme c'est à Ève de jouer, il suffit qu'elle joue vers cet état.

\item $\phi = \mbox{AX } \psi$. Donc $succ(q_0, \phi) = \{ (q, \psi) \mid (q_0,q) \in T \} $ et c'est à Adam de jouer. Comme $q_0 \vDash \phi$, pour tous les successeurs $q$ de $q_0$, on a que $q \vDash \psi$. Donc, par hypothèse de récurrence, il existe une stratégie gagnante pour Ève pour chacun des $(q, \psi)$. C'est à Adam de jouer, mais quelque soit son choix, Ève a une stratégie gagnante. Donc Ève a une stratégie gagnante.

\item $\phi = \mbox{E } \psi_1 \mbox{ U } \psi_2$ ou $\phi = \mbox{E } \psi_1 \mbox{ W } \psi_2$. Donc $succ(q_0,\phi) = \{ (q_0,\psi_2), (q_0,\psi_1 \land \mbox{EX } \phi) \}$ et c'est à Ève de jouer. $q_0 \vDash \phi$, il y a donc deux possiblités :
\begin{itemize}
	\item Soit $q_0 \vDash \psi_2$ et, par hypothèse de récurrence, alors Ève a une stratégie gagnante à partir de $(q_0,\psi_2)$. Il lui suffit donc de choisir cette transition.
	\item Soit $q_0 \vDash \psi_1 \land \mbox{EX } \phi$.
	Par récurrence, Ève a une stratégie gagnante à partir de $(q_0,\psi_1)$.
	Montrons qu'elle en a aussi une à partir de $(q_0,\mbox{EX } \phi)$.
	Il y a deux cas:
	\begin{itemize}
		\item Soit il existe un chemin à partir de $q_0$ tel que $\psi_1$ est vrai jusqu'à ce que $\psi_2$ le soit. Par récurrence sur la taille de ce chemin, montrons qu'Ève a une stratégie gagnante.\\
		S'il est de taille 1, cela veut dire qu'il existe un successeur $q$ de $q_0$ tel que $\psi_2$ soit vrai, donc par récurrence, Ève a une stratégie gagnante à partir de $q$, il suffit donc à Ève de prendre le chemin $(q_0,q)$.\\
		Sinon, Ève doit prendre le successeur $q$ de $q_0$ le long du chemin tel que $\psi_1$ soit vraie, donc $q \vDash \psi_1 \land \mbox{EX } \phi$, et donc, par récurrence, Ève a une stratégie gagnante à partir de $(q_0,\mbox{EX } \phi)$.
		\item Soit $\phi = \mbox{E } \psi_1 \mbox{ W } \psi_2$ et il y a un chemin à partir de $q_0$ se finissant par un cycle tel que $\psi_1$ soit vrai sur tous les états de ce chemin. La stratégie d'Ève est donc simplement de suivre ce cycle à l'infini, car, pour tout $q$, $c(q,\mbox{E } \psi_1 \mbox{ W } \psi_2) = 0 \mod 2$.
	\end{itemize}
\end{itemize}
Donc Ève a une stratégie gagnante dans tous les cas.

\item $\phi = \mbox{A } \psi_1 \mbox{ U } \psi_2$ ou $\phi = \mbox{A } \psi_1 \mbox{ W } \psi_2$. Donc $succ(q_0,\phi) = \{ (q_0,\psi_2), (q_0,\psi_1 \land \mbox{AX } \phi) \}$ et c'est à Ève de jouer. $q_0 \vDash \phi$, il y a donc deux possibilités :
\begin{itemize}
	\item Soit $q_0 \vDash \psi_2$ et, par hypothèse de récurrence, alors Ève a une stratégie gagnante à partir de $(q_0,\psi_2)$.
	\item Soit $q_0 \vDash \psi_1 \land \vDash \mbox{AX } \phi$. Donc, $q_0 \vDash \psi_1$ et par hypothèse de récurrence, Ève a une stratégie gagnante à partir de $(q_0,\psi_1)$. De plus, $q_0 \vDash \mbox{AX } \phi$, donc pour tous les successeurs $q$ de $q_0$, $q \vDash \phi$.
	\begin{itemize}
	\item Si $\phi = \mbox{A } \psi_1 \mbox{ U } \psi_2$, c'est donc que quel que soit le chemin que l'on prend depuis un $q$ comme ci-dessus, on atteint un $q'$ tel que $q' \vDash \psi_2$. Par récurrence, Ève a une stratégie gagnante depuis $(q', \psi_2)$, donc elle en a une depuis $(q_0,\phi)$.
	\item Si $\phi = \mbox{A } \psi_1 \mbox{ W } \psi_2$, c'est donc que quel que soit le chemin que l'on prend depuis un $q$ comme ci-dessus, soit on atteint un $q'$ tel que $q' \vDash \psi_2$, soit on visite infiniment souvent des états $q_i$ tels que $q_i \vDash \psi_1$. Si on est dans le premier cas, il suffit de reprendre la démonstration donnée ci-dessus. Sinon la stratégie d'Ève est simplement de suivre ce cycle à l'infini, car, pour tout $q$, $c(q,\mbox{E } \psi_1 \mbox{ W } \psi_2) = 0 \mod 2$.
	\end{itemize}
\end{itemize}
Donc Ève a une stratégie gagnante dans tous les cas.
\end{itemize}

\subsection{Résolution des jeux faibles de parité}
\label{algofpg}
Soit $\gamma$ un joueur (Ève ou Adam). On notera par la suite $\overline{\gamma}$ l'autre joueur.

\subsubsection{Prémisses}
Nous allons utiliser quelques méthodes que nous n'allons pas définir:
\begin{itemize}
	\item $compute\_cfc$ renvoie la liste des composantes fortement connexes (ou CFC) d'un graphe orienté.
	\item $topological\_sort$ effectue un tri topologique des CFC, de sorte qu'une CFC $V_i$ se trouve avant une CFC $V_j$ si et seulement s'il n'y a pas de chemin entre un sommet de $V_j$ et un de $V_i$. 
	\item $reverse(l)$ renvoie une liste avec les éléments de $l$ dans l'ordre inverse.
	\item $get\_winner(V_i)$ renvoie:
	\begin{itemize}
		\item Si la CFC n'a qu'un état, le joueur qui joue cet état
		\item Sinon, le joueur qui gagne si on boucle dans la CFC.
	\end{itemize}
\end{itemize}
Notons que toutes ces méthodes peuvent être implémentées en un temps au plus linéaire en le nombre de sommets et d'arêtes.

\subsubsection{Algorithme de résolution du jeu}
Voici maintenant un algorithme $solve\_weak\_game$ permettant de connaitre le gagnant d'un jeu faible de parité.

\begin{algorithm}
	\caption{$solve\_weak\_game$}
\begin{algorithmic}[1]
	\REQUIRE $(q_0, V, E)$ \COMMENT{l'état de départ et le graphe du jeu}
	\STATE $CFC \leftarrow compute\_cfc(V, E)$
	\STATE $CFC \leftarrow topological\_sort(CFC)$
	\STATE $CFC \leftarrow reverse(CFC)$
	\STATE $winners \leftarrow new\_array(size(V), \bot)$
	\FOR { $V_k \in CFC$ }
		\STATE $\gamma \leftarrow get\_winner(V_k)$
		\STATE $propagate\_opposite(\gamma, V_k, E, winners)$
		\FOR { $x \in V_k$ }
			\IF { $winners[x] = \bot$ }
				\STATE $winners[x] = \gamma$
			\ENDIF
		\ENDFOR
	\ENDFOR
	\RETURN $winners[q_0]$
\end{algorithmic}
\end{algorithm}

\begin{algorithm}
\caption{$propagate\_opposite$}

\begin{algorithmic}[1]
	\REQUIRE $(\gamma, V, E, winners)$
	\STATE $CHANGED \leftarrow FALSE$
	\FOR{$x \in V$}
		\IF{$winners[x] = \bot$}
			\STATE $EXI \leftarrow FALSE$
			\STATE $ALL \leftarrow TRUE$
			\FOR {$(x,y) \in E$}
				\IF{$winners[y] = \overline{\gamma}$}
					\STATE $EXI \leftarrow TRUE$
				\ELSE
					\STATE $ALL \leftarrow FALSE$
				\ENDIF
			\ENDFOR
			\IF{$(get\_player(x) = \overline{\gamma} \land EXI) \lor ALL $}
				\STATE $CHANGED \leftarrow TRUE$
				\STATE $winners[x] = \overline{\gamma}$
				\STATE BREAK \COMMENT{Non nécessaire, mais simplifie la preuve}
			\ENDIF
		\ENDIF
	\ENDFOR
	\IF{$CHANGED$}
		\STATE $propagate\_opposite(\gamma, V, E, winners)$
	\ENDIF
	\RETURN
\end{algorithmic}
\end{algorithm}

\subsubsection{Terminaison}
Montrons que $solve\_weak\_game$ termine. Cela revient à montrer que $propagate\_opposite$ termine. Or la fonction ne termine pas immédiatement seulement si un appel récursif a lieu, ce qui est le cas uniquement si la variable $CHANGED$ est mise à $TRUE$. Cependant cela n'arrive que si l'on assigne un gagnant à un état $y$ \emph{qui n'en avait pas avant}. L'ensemble des états étant finis, les appels récursifs se terminent forcément.

\subsection{Correction}
Montrons que $solve\_weak\_game$ est correcte, c'est-à-dire que $solve\_weak\_game(q_0,V,E) = \gamma$ si et seulement s'il existe une stratégie gagnante pour $\gamma$.

\paragraph{$\implies$}
Montrons en fait par récurrence sur $k$ qu'à la fin du traitement d'une CFC $V_k$, on a correctement attribué le gagnant de chaque état. Il suffit pour ça de montrer $propagate\_opposite$ va effectivement attribuer les états remportés par $\overline{\gamma}$, et uniquement ceux-ci.
\subparagraph{Si $k = 1$:} on traite la première CFC. Remarquons que, comme les $V_k$ ont été ordonnées par ordre topologique inversé, les successeurs de tous les états de $V_1$  sont dans $V_1$.\\
Si $V_1$ est réduite à un seul élément, c'est à $\gamma$ de jouer, mais il est bloqué, donc $\overline{\gamma}$ gagne cet état et $propagate\_opposite$ détecte bien ce cas (car pour l'unique état $x$ sans successeurs, $ALL$ vaudra $TRUE$).\\
Sinon, $\gamma$ est le joueur qui gagne en restant dans la CFC, et quel que soit le joueur ayant le trait, il est obligé de rester dans la CFC, donc $\gamma$ gagne. $propagate\_opposite$ détecte bien ce cas, aucun état n'est gagnant pour $\overline{\gamma}$ et tous les états ont un successeur, donc pour tous les états $x \in V_1$, $ALL = FALSE$ et $EXI = FALSE$, donc $winners[x]$ ne sera pas initialisé.

\subparagraph{Sinon:} on suppose le résultat vrai pour toutes les CFC $V_i$, $i < k$.
\\
Si $x \in V_k$ est un état gagnant pour $\overline{\gamma}$ c'est que :
\begin{enumerate}
	\item Soit c'est à $\gamma$ de jouer et il est bloqué ce qui est le cas si $x$ n'a pas de successeur, ou tous ses successeurs sont gagnants pour $\overline{\gamma}$.
	\item Soit c'est à $\overline{\gamma}$ de jouer et il existe un successeur de $x$ qui est gagnant pour $\overline{\gamma}$.
\end{enumerate}
Montrons que $winners[x] = \overline{\gamma}$.\\
Si $x$ n'a pas de successeur ou que la CFC est close, on peut reprendre la démonstration donnée pour $V_1$.\\
Sinon, c'est qu'il existe un état $y \in V_i$, $ i < k$ accessible depuis $x$ tel que tous les états sur le chemin $\pi$ entre $x$ et $y$ soient gagnants pour $\overline{\gamma}$ (car $\overline{\gamma}$ perd s'il boucle dans $V_k$). On a par hypothèse de récurrence, $winners[y] = \overline{\gamma}$.\\
Choisissons un $y$ qui a ces propriétés et tel que $\pi$ soit le plus grand possible et sans boucle:
\begin{itemize}
	\item Si $\pi$ est un chemin de taille 1:
	\begin{itemize}
		\item Si on est dans le cas 1, il est clair que $EXI$ vaudra $TRUE$ (car $(x,y) \in E$), et donc que $winners[x] = \overline{\gamma}$.
		\item Si on est dans le cas 2, tous les successeurs de $x$ sont gagnants pour $\overline{\gamma}$ et ils se trouvent tous des CFC d'indices plus petites (sinon cela contredirait l'hypothèse de maximalité sur la taille de $\pi$). Donc pour chaque successeur $y$ de $x$, $winners[y] = \overline{\gamma}$ et donc $ALL$ vaudra $TRUE$, et donc $winners[x] = \overline{\gamma}$.
	\end{itemize}
	\item Si $\pi$ est un chemin de taille $n$
		\begin{itemize}
		\item Si on est dans le cas 1, $x$ a un successeur $y$ dans $\pi$ et donc par hypothèse de récurrence, $winners[y] = \overline{\gamma}$ et donc $EXI$ vaudra $TRUE$ et $winners[x] = \overline{\gamma}$.
		\item Si on est dans le cas 2, tous les successeurs de $x$ sont gagnants pour $\overline{\gamma}$. Mais ils se trouvent soit dans une CFC d'indice plus petites, soit le long d'un chemin gagnant pour $\overline{\gamma}$ de taille plus petite que $\pi$, et donc par récurrence, $ALL$ vaudra $TRUE$ et donc $winners[x] = \overline{\gamma}$.
	\end{itemize}
\end{itemize}
Réciproquement, si $winners[x] = \overline{\gamma}$ à la fin du traitement par $propagate\_opposite$, montrons que $x$ est gagnant pour $\overline{\gamma}$. Montrons ce résultat par récurrence sur l'indice de l'appel récursif $propagate\_opposite$ où $winners[x]$ a été mis à $\overline{\gamma}$. Notons tout d'abord que si $winners[x]$ a été mis à $\overline{\gamma}$ c'est que lors du traitement de $x$, la condition $(get\_player(x) = \overline{\gamma} \land EXI) \lor ALL $ était vraie.\\
Si on est dans le premier appel à $propagate\_opposite$
\begin{itemize}
	\item Si $get\_player(x) = \overline{\gamma} \land EXI$ était vrai, c'est qu'il existait un successeur $y$ de $x$ tel que $winners[y] = \overline{\gamma}$. Mais comme on est dans le premier appel à $propagate\_opposite$ c'est que $y$ appartenait à une CFC d'indice plus basse. Donc, par hypothèse de récurrence, $y$ est gagnant pour $\overline{\gamma}$. Donc $x$ est gagnant pour $\overline{\gamma}$
	\item Si $ALL$ était vrai, c'est que pour tous les successeurs $y$ de $x$, $winners[y] = \overline{\gamma}$, or on est dans le premier appel récursif à $propagate\_opposite$, donc tous les $y$ appartiennent à des CFC de plus petit indice, et donc par un argument similaire au premier point $x$ est gagnant pour $\overline{\gamma}$.
\end{itemize}
Si on est dans le $n$-ième appel récursif à $propagate\_opposite$,
\begin{itemize}
	\item Si $get\_player(x) = \overline{\gamma} \land EXI$ était vrai, c'est qu'il existait un successeur $y$ de $x$ tel que $winners[y] = \overline{\gamma}$. Donc soit $y$ a était mis à $\overline{\gamma}$ par un appel à $propagate\_opposite$ antérieur, ou car $y$ est dans une CFC d'indice plus petite. Dans tous les cas, on peut conclure que $x$ est gagnant pour $\overline{\gamma}$
	\item Si $ALL$ était vrai, c'est que pour tous les successeurs $y$ de $x$, $winners[y] = \overline{\gamma}$, donc tous les $y$  ont été mis à $\overline{\gamma}$ par un appel à $propagate\_opposite$ antérieur, ou car $y$ est dans une CFC d'indice plus petite. Dans tous les cas, on peut conclure que $x$ est gagnant pour $\overline{\gamma}$
\end{itemize}

Donc par récurrence, $x$ est gagnant  pour $\overline{\gamma}$.

\paragraph{$\Longleftarrow$}
Par contraposée, montrons que si $solve\_weak\_game(q_0,V,E) \neq \gamma$ alors il n'y a pas de stratégie gagnante pour $\gamma$.\\
En effet, les jeux faibles de parité sont déterminés, donc si $solve\_weak\_game(q_0,V,E) \neq \gamma$, cela implique que $solve\_weak\_game(q_0,V,E) = \overline{\gamma}$, et d'après l'implication précédente, qu'il existe une stratégie gagnante pour $\overline{\gamma}$, donc il n'en existe pas pour $\gamma$.

\section{Implémentation}
\subsection{Introduction}
Nous avons implémenté l'algorithme décrit Section \ref{algofpg} en OCaml, le code est disponible à l'adresse suivante: \url{https://github.com/nobrakal/ctl/}.

\subsection{Modèle checking de CTL: marquage}
Nous avons aussi, pour comparaison, implémenté une autre méthode de modèle checking de CTL, basée sur le parcours et le marquage de la structure de Kripke. Notre implémentation est une traduction fonctionnelle de l'algorithme décrit par Philippe Schnoebelen et al. \cite{verif}.

\bibliographystyle{plain}
\bibliography{biblio}
\end{document}
